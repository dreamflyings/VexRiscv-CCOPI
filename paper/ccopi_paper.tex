
\documentclass[twoside,twocolumn]{article}

% ------
% Fonts and typesetting settings
\usepackage[sc]{mathpazo}
\usepackage[T1]{fontenc}
\linespread{1.05} % Palatino needs more space between lines
\usepackage{microtype}
\usepackage{amsmath}

% ------
% Page layout
\usepackage[hmarginratio=1:1,top=32mm,bottom=40mm,columnsep=20pt]{geometry}
\usepackage[font=it]{caption}
\usepackage{paralist}
%\usepackage{multicol}

\usepackage{graphicx}


% ------
% Lettrines
\usepackage{lettrine}

% ------
% Abstract
\usepackage{abstract}
	\renewcommand{\abstractnamefont}{\normalfont\bfseries}
	\renewcommand{\abstracttextfont}{\normalfont\small\itshape}


% ------
% Titling (section/subsection)
\usepackage{titlesec}
\renewcommand\thesection{\Roman{section}}
\titleformat{\section}[block]{\large\scshape\centering}{\thesection.}{1em}{}


% ------
% Header/footer
\usepackage{fancyhdr}
	\pagestyle{fancy}
	\fancyhead{}
	\fancyfoot{}
	\fancyhead[C]{Fachseminar ``Machine Learning'' $\bullet$ Wintersemester 17/18 $\bullet$ Prof. Dr. Steffen Reith}
	\fancyfoot[RO,LE]{\thepage}


\usepackage{color}

% ------
% Maketitle metadata
\title{\vspace{-7mm}%
	\fontsize{24pt}{10pt}\selectfont
	\textbf{CCOPI: Implementing a custom coprocessor interface for
    VexRiscv}
	}	
\author{%
	\large
	\textsc{Jens Nazarenus, Dominik Swierzy} \\[2mm]
	\normalsize	Studiengang Master Informatik - SSMT, Hochschule RheinMain \\
    \normalsize	\{\;jens.nazarenus|dominik.swierzy\}@hs-rm.de
	%\vspace{-5mm}
	}
\date{\today}

\usepackage[utf8]{inputenc}

\usepackage{hyperref}


%%%%%%%%%%%%%%%%%%%%%%%%
\begin{document}

\maketitle
\thispagestyle{fancy}

\section{Abstract}
In this paper we introduce CCOPI, a custom coprocessor interface for the
RISC-V implementation VexRiscv. CCOPI as well as VexRiscv is written in
the hardware description language SpinalHDL. The interface is responsible 
for the communication between the coprocessor and the core CPU pipeline of 
VexRiscv and thus helps hardware developers in the designing process of 
a coprocessor with a custom instruction-set extension.

CCOPI uses the fexibility of the RISC-V implementation VexRiscv to create 
the interface. This paper also shows how VexRiscv is designed particulary with 
regard to modifications and custom extensions. 

\section{Introduction}

\bibliographystyle{unsrt}
\bibliography{literature}


\end{document}
